%TODO:https://www.zhihu.com/question/20500497
\chapter{信号与系统}
不管是汽车发动机,还是电路,都可以抽象成一个系统。一个系统,有输入信号,也有输出信号,输入和输出之间满足一定的关系。信号与系统,就是研究系统的输入和输出之间的变化规律的科学。
\section{连续系统与离散系统}
\subsection{单输入单输出连续系统}
系统有一个输入信号,表达为$x(t)$,一个输出信号,表达为$y(t)$输入和输出信号之间存在一定的关系,表达为:
\begin{equation}
	y(t)=H{x(t)}
\end{equation}

\begin{quote}
	从公式上看函数与系统非常相似,不同之处仅在于系统的输入、输出信号都是时间变量t的函数。

	函数表达的是因变量y随着自变量x的变化规律,而x和y本身是静态的。在系统当中,因为有时间变量的存在,输入和输出都是随着时间变化的,其表达的相互关系不只是某一个时刻的关系,而是一段时间内的关系。

	更具体一点地说,在某一时刻系统的输出值,不仅与该时刻的输入有关,还可能与历史的输入有关。
\end{quote}


\subsection{单输入单输出离散系统}
输入信号$x[n]$,经过离散系统的变换后,输出信号为$y[n]$,其中$n$为整数,可以记作:
\begin{equation}
	y[n]=H{x[n]}
\end{equation}

\subsection{连续离散与模拟数字}
\subsubsection{连续和离散}
连续和离散,是从自变量的角度去区别的。
\begin{enumerate}
	\item 连续信号:以连续时间变量t为自变量的函数;
	\item 连续系统:以连续信号为输入输出的系统就是连续系统;
	\item 离散信号:可以由连续信号在离散的时间点上取样获得;

	      一个均匀采样的离散信号表达为:
	      \begin{equation}
		      x[n]=x(n\cdot \Delta t)
	      \end{equation}

	      这里采用方括号表示离散信号,圆括号表示连续信号。

	      $n$为整数,可以为正,可以为负,可以为零。
	\item 离散系统:以离散信号为输入输出的系统就是离散系统。
\end{enumerate}
\subsubsection{模拟和数字}
模拟和数字,是从因变量的取值上去区别的。
\begin{enumerate}
	\item 模拟信号:取值范围是实数域;
	\item 数字信号:用有限位的O和1来表达一个值,有一定的表示精度。
	\item 连续信号经过采样成为离散信号,这个时候还是模拟的,再经过A/D转换器后用有限的位数去表达从而成为数字信号。
	\item 基本上所有的离散信号都会进行数字化处理。
\end{enumerate}
\section{线性系统}
线性系统是指具有下面两个性质的系统:
\begin{enumerate}
	\item 叠加性
	      系统对两个信号的和的输出,等于两个信号输出的和;
	\item 数乘性
	      如果输入信号放大$\alpha$倍,则输出信号也放大$\alpha$倍。
\end{enumerate}
\subsection{连续系统}
\begin{equation}
	H[x_1(t)+x_2(t)]=H[x_1(t)]+H[x_2(t)]
\end{equation}
\begin{equation}
	H[\alpha\cdot x(t)]=\alpha\cdot H[x(t)]
\end{equation}
\subsection{离散系统}
\begin{equation}
	H\{x_1[n]+x_2[n]\}=H\{x_1[n]\}+H\{x_2[n]\}
\end{equation}
\begin{equation}
	H\{\alpha\cdot x[n]\}=\alpha\cdot H\{x[n]\}
\end{equation}
\section{时/移不变系统}
\subsection{时不变系统}
时不变系统是指具有下面性质的连续系统:

如果$y(t)=H\{x(t)\}$,则$y(t-\tau)=H\{x(t-\tau)\}$。

意思是说,如果输入信号延时了一段时间$\tau$那么输出信号也延时相同的时间。

在函数的概念里面,函数是固定的,这种变量的代换可以随便做。然而对于一个系统,它对输入信号的响应可能随着时间的变化而变化。
\subsection{移不变系统}
与时不变系统对应,在离散系统当中叫移不变系统。因为离散系统已经没有了时间的变量,而是用一个序号作为自变量,因此叫作移(动)不变系统。

需要满足下面的要求:

如果$y[n]=H\{x[n]\}$,则$y[n+m]=H\{x[n+m]\}$。

\section{线性系统对激励的响应}
\begin{quote}
	我们也把输入信号叫作激励,把输出信号叫作响应。
\end{quote}
\subsection{离散系统对激励的响应}
\subsubsection{离散$\delta$信号}
\begin{equation}
	\delta[n]=
	\left\{
	\begin{aligned}
		1  \qquad & 当n=0     \\
		0  \qquad & 当n\neq 0
	\end{aligned}
	\right.
\end{equation}

在模拟领域,一般用信号或者函数的名称;而在数字领域,一般称离散信号或者序列,所以上面表达的信号叫作离散冲激序列。
\subsubsection{离散卷积}
任何一个离散信号$x[n]$都可以表达成如下形式:
\begin{equation}
	x[n]=\sum\limits_{k=-\infty}^\infty\ x[k]\cdot \delta[n-k]
\end{equation}

如果一个离散系统$H{\cdot}$是一个线性系统,根据其性质,则可得:
\begin{equation}
	\begin{aligned}
		  & y[n]=H\{x[n]\}                                             \\
		= & H\{\sum\limits_{k=-\infty}^\infty\ x[k]\cdot \delta[n-k]\} \\
		= & \sum\limits_{k=-\infty}^\infty\ x[k]\cdot H\{\delta[n-k]\}
	\end{aligned}
\end{equation}

因为$n$才是“自变量”,所以这里把$x[k]$视作常数。

我么做如下定义:
\begin{equation}
	h[n]=H\{\delta[n]\}
\end{equation}

如果该系统进一步是一个移不变系统,则有:
\begin{equation}
	h[n-k]=H\{\delta[n-k]\}
\end{equation}

代入到前面的公式得到:
\begin{equation}
	y[n]=\sum\limits_{k=-\infty}^\infty\ x[k]\cdot h[n-k]
\end{equation}

这个公式就是著名的离散卷积,也记作:
\begin{equation}
	y[n]=x[n]*h[n]
\end{equation}

$h[n]$叫作系统的冲激响应,反映了系统的特性,而系统的输出是输入和冲激响应的卷积。

\begin{quote}
	移不变系统并不是反折不变的!!!!

	\begin{align*}
		\because   & \quad \delta[-n]=\delta[n]                     \\
		\therefore & \quad H\{\delta[k-n]\}=H\{\delta[n-k]\}=h[n-k] \\
		\because   & \quad h[k-n]\neq h[n-k]                        \\
		\therefore & \quad H\{\delta[k-n]\}\neq h[k-n]
	\end{align*}
\end{quote}
\subsubsection{因果系统}
因果系统就是,如果没有信号输入,则系统就没有输出,即:
\begin{equation}
	h[n]=0\qquad (n< 0)
\end{equation}

因此,如果是因果系统,$n$时刻系统的输出信号只与该时刻之前的输入信号有关,而和该时刻之后的输入信号无关,因此,求和的上限可以取为$n$:
\begin{equation}
	y[n]=\sum\limits_{k-\infty}^\infty\ x[k]\cdot h[n-k]
\end{equation}

在离散卷积的推导过程当中,用到了线性系统和移不变的条件,这是卷积成立的前提。

\subsection{连续系统对激励的响应}
\subsubsection{连续$\delta$信号}
\begin{equation}
	\left\{
	\begin{aligned}
		\sum_{-\infty}^{\infty}\ \delta(t)\ dt=1 \\
		\delta(t)=0\qquad t\neq 0
	\end{aligned}
	\right.
\end{equation}
也就是说,冲激函数零点的值为无穷,除此之外其他位置的值都是零,而在无穷时间上的积分为单位$1$。
\subsubsection{连续卷积}
\begin{align*}
	  & \sum\limits_{-\infty}^{\infty}\ x(t)\cdot \delta(t-t_0)\ dt   \\
	= & \sum\limits_{-\infty}^{\infty}\ x(t_0)\cdot \delta(t-t_0)\ dt \\
	= & x(t_0)
\end{align*}

$\delta(t-t_0)$是位于$t_0$时刻的一个冲激,信号$x(t)$与之相乘并积分后,得到信号在$t_0$的值。

把上式做变量替换,$t=\tau$,$t_0=t$,成为:
\begin{equation}
	x(t)=\int\limits_{-\infty}^\infty\ x(\tau)\cdot \delta(t-\tau)\ d\tau
\end{equation}

积分也就是求和的意思。也就是说,任何一个函数$x(t)$,都可以写成一系列不同时移的冲激函数$\delta(t-\tau)$的加权和的形式。

\begin{align*}
	\because   & h(t)=H\{x(t)\}                                                   \\
	\therefore & y(t)=H\{x(t)\}                                                   \\
	           & =H\{\int_{-\infty}^\infty\ x(\tau)\cdot \delta(t-\tau)\ d\tau \} \\
	           & =\int_{-\infty}^\infty\ x(\tau)\cdot H\{\delta(t-\tau)\}\ d\tau  \\
	           & =\int_{-\infty}^\infty\ x(\tau)\cdot h(t-\tau)\ d\tau
\end{align*}

这个积分就是著名的卷积公式,也记作:
\begin{equation}
	y(t)=x(t)*h(t)
\end{equation}
\subsubsection{因果系统}
因果系统就是如果没有信号输入,系统就没有输出,也就是:
\begin{equation}
	h(t)=0\qquad t<0
\end{equation}

因此,如果是因果系统,$t$时刻系统的输出信号只与该时刻前的输入信号有关,
而和该时刻之后的输入信号无关,因此,积分的上限可以取为$t$:
\begin{equation}
	y(t)=\int_{-\infty}^t\ x(\tau)\cdot h(t-\tau)\ d\tau
\end{equation}

有了输入信号和冲激响应,线性系统的输出就被完全决定了。线性系统对冲激信号的响应,完全反映了系统的特性。

\begin{quote}
	\begin{enumerate}
		\item $y(t)$是$t$时刻系统的输出信号;
		\item 它是由$t$时刻之前的输入信号的无数的小冲激脉冲引起的,因此积分上限为$t$;
		\item 积分变量用$\tau$表示;
		\item 时刻$\tau$处的一个冲激脉冲,强度为$x(\tau)\ d\tau$;
		\item 它引起的系统的输出为$x(\tau)\ d\tau\cdot h(t-\tau)$。$t-\tau$的意思是,把冲激响应的波形右移了$\tau$;
		\item 积分起来就行了。
	\end{enumerate}
\end{quote}
\section{卷积的性质}
\subsection{交换率}
\begin{align*}
	x_1(t)*x_2(t) & =x_2(t)*x_1(t) \\
	x_1[n]*x_2[n] & =x_2[n]*x_1[n]
\end{align*}

因为:
\begin{equation}
	x_1(t)*x_2(t)=\int_{-\infty}^\infty\ x_1(\tau)\cdot x_2(t-\tau)\ d\tau
\end{equation}

把积分变量$\tau$更换为$t-\lambda$,则:
\begin{equation}
	x_1(t)*x_2(t)=\int_{-\infty}^\infty\ x_2(\lambda)\cdot x_1(t-\lambda)\ d\lambda=x_2(t)*x_1(t)
\end{equation}
\subsection{分配率}
\begin{align*}
	x_1(t)*[x_2(t)+x_3(t)]=x_1(t)*x_2(t)+x_1(t)*x_3(t) \\
	x_1[n]*[x_2[n]+x_3[n]]=x_1[n]*x_2[n]+x_1[n]*x_3[n]
\end{align*}
\subsection{结合率}
\begin{align*}
	\left[x_{1}(t)*x_{2}(t)\right]*x_{3}(t)=x_{1}(t)*\left[x_{2}(t)*x_{3}(t)\right] \\
	\left[x_{1}[n]*x_{2}[n]\right]*x_{3}[n]=x_{1}[n]*\left[x_{2}[n]*x_{3}[n]\right]
\end{align*}
\subsection{与冲激函数卷积}
\begin{align*}
	\delta(t)*x(t)=x(t)*\delta(t)=x(t) \\
	\delta[n]*x[n]=x[n]*\delta[n]=x[n]
\end{align*}
一个函数与冲激函数的卷积是其本身。
